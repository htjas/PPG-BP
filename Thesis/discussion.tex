The general objective of this study was to ascertain the feasibility of employing the PPG technology for the estimation of BP in ICU patients.
This goal was realized by undertaking three main tasks outlined in subsection~\ref{subsec:tasks_objectives}.

% Task 1
\subsection{Data Fetching and Filtering}
\label{subsec:data-fetching-and-filtering}

The initial task of the investigation involved identifying an effective strategy for fetching and filtering data from accessible ICU data repositories.
This objective was successfully accomplished by utilizing the publicly accessible datasets of MIMIC-III and MIMIC-IV,
employing fetching and filtering methodologies to consistently generate physiologically satisfactory ABP and PPG visual signals.
These procedures are detailed in the methods section~\ref{subsubsec:filtering}.
Despite the thorough exploration of various data processing methods, there remains potential for further experimentation and the discovery of even more optimal techniques
to produce discrete and optimally representative physiological pulse waveform graphics.

One potential improvement, not within the scope of this study due to the substantial volume of samples analysed, involves the manual filtering of waveform graphics.
A straightforward method would entail reviewing all segments to determine if they present a typical healthy pulse wave.
Segments failing to meet this criterion would then be manually excluded from subsequent procedures.
This approach could be advantageous in identifying erroneous PPG signals that may have bypassed automated exception handling, thereby potentially negatively affecting the overall dataset quality.

The study encountered several constraints, particularly within this task, such as the absence of details regarding the devices utilized,
their calibration, and accuracy, alongside the diverse sampling rates among devices and datasets.
The MIMIC-III dataset was captured at a rate of 125 Hz, while MIMIC-IV was at 62.4725 Hz, potentially resulting in discrepancies.
Additionally, the specific devices employed for PPG and ABP recording were not specified, making it exceedingly challenging to draw consistent conclusions regarding the accuracy and fidelity of the signals.

% Task 2
\subsection{PPG Feature Extraction}
\label{subsec:ppg-feature-extraction}

The successful execution of the second task, which involved developing a consistent algorithm for extracting key features from PPG signals,
is evidenced by the average number of data points extracted from each array, as detailed in the results section~\ref{subsec:results_data}.
Notably, the metric obtained from the MIMIC-III dataset surpassed that of MIMIC-IV by a considerable margin.
This indicates the efficacy of the signal processing algorithms in extracting a greater number of values from the MIMIC-III DB.
The differences observed in the efficiency of signal processing between the MIMIC-III and MIMIC-IV DBs raise considerations about data quality.
It is plausible to infer that the signal quality was superior in MIMIC-III, given its status as an older and more refined DB\@.
This conjecture aligns with the methodology employed, as there was no limitation on the retrieval of records from a single study or subject in MIMIC-IV\@.
Consequently, a substantial number of segments might have been discarded during the signal processing exception handling phase (as documented in~\ref{subsubsec:beats}) if they did not exhibit a coherent signal.
In contrast, the data fetching approach from MIMIC-III adhered to a set pattern by imposing a limit of 100 records from a single study, potentially resulting in a more consistent quality of signals.

One significant caveat lies in the reliance of three features, specifically Delta Time, Delta Area, and Resistive Index, on the precise identification of the dicrotic notch.
This constraint significantly affects the quantity of values that can be extracted, thereby impeding the model's capacity to fully utilize the dataset.

Additionally, the feature importance plots offer a pivotal insight and significant outcome of the study, showcasing the features that are correlated and have the most substantial impact on precise BP prediction.
Through the iterative evaluation process outlined in the methods section~\ref{subsubsec:feature_importance}, the feature importance values underwent a systematic analysis,
revealing their significant influence on the models' predictive efficacy.
The plots, available in the Appendix~\ref{subsec:plots_fi}, provide a visual depiction of all PPG features employed for BP prediction, arranged in descending order,
thereby displaying the features' roles in either enhancing or diminishing the predictive capability of various models.
The findings unveiled a general classification of features into three groups based on their average importance values.
Notably, within the category of top performers, features such as Diastolic Time, Resistive Index, and Normalized Power at Peak emerged prominently,
demonstrating substantial influence in enhancing the models' accuracy and exhibiting a stronger correlation with BP compared to other features.
Consequently, as mentioned in the preceding paragraph, these features reliant on the dicrotic notch could not be excluded, as doing so would have negatively impacted prediction performance,
particularly since the Resistive Index emerged as one of the highest-scoring features in the importance hierarchy.

Another noteworthy consideration is the absence of feature reduction, with the models being trained on a comprehensive set of 34 features.
The decision against implementing feature reduction stemmed from the observation that each feature made a positive contribution to each model's performance,
as evidenced by the feature importance plots (refer to~\ref{subsec:plots_fi}).
In comparison, previous studies have often utilized a smaller number of features, typically ranging from around 20 to 5~\cite{el-hajjDeepLearningModels2021, charltonAssessingHemodynamicsPhotoplethysmogram2022, yilmazNocturnalBloodPressure2023}.
Therefore, the extensive size of the training features could potentially result in reduced predictive accuracy,
particularly if the model encounters difficulties in discerning meaningful patterns or relationships within the dataset.
This challenge may be exploited by the presence of redundant or noisy features, which might introduce unnecessary complexity into the learning process.

% Task 3
\subsection{ML Model Development and Testing}
\label{subsec:ml-model-development-and-testing}

Finally, the third objective entailed the development and evaluation of diverse ML models utilizing the extracted features to predict BP reliably from PPG signals.
In accordance with the collaborative 2018 statement issued by the American National Standards of the Association for the Advancement of Medical Instrumentation (AAMI), the European Society of Hypertension (ESH), and the International Organization for Standardization (ISO),
the recommended average deviation and its standard deviation in NIBP measurements should ideally range within 5 ± 8 mmHg compared to a reference BP, based on a minimum of 85 patient assessments.
A further consensus was reached, that a BP measuring device is considered acceptable if its estimated probability of a tolerable error ($\leq$10 mmHg) is at least 85\%~\cite{stergiouUniversalStandardValidation2018}.
Evidently, the number of patient assessments conducted in this investigation far exceeds the stipulated minimum of 85, encompassing both the training/testing and validation datasets (see available subjects in Figure~\ref{fig:data_flow} for reference).
However, the accuracy metrics did not meet the desired levels.

While the performance of advanced NN models was acceptable, it did not exhibit exceptional results.
The RF model showed promise with a low test loss, but faced challenges evidenced by a higher validation loss compared to LSTMs and GRUs.
Notably and surprisingly, the MLP model emerged as the most effective in minimizing average variation, achieving an MAE of 8.328 ± 5.814 mmHg for estimating DBP\@.
The recorded average and standard deviation did not reach the ideal goal range of 5 ± 8 mmHg, as specified by the AAMI, ESH and ISO\@.
Nonetheless, it did attain an acceptable error level, remaining below 10 mmHg.

Another intriguing finding was the performance of the MLP, one of the less complex ML models developed in this project.
Although by a slight margin, it surpassed both unidirectional and bidirectional LSTMs and GRUs, with weight adjustment showing no improvement.
Conversely, weight adjustment resulted in even poorer performance for the MLP model.

The simplest model employed was the LR, which surpassed the author's expectations by ranking at the pinnacle of validation measures.
This could be attributed to its adeptness in optimizing the linear relationships between PPG and ABP features,
as evidenced by the minimal variance in testing and validation metrics (less than 0.5 mmHg difference observed across both weighted and unweighted versions).

An analysis of the average validation MAE across distinct BP measurements revealed significant divergences, particularly with SBP displaying the highest margin of error.
Conversely, DBP demonstrated a substantial decrease, nearly halving the MAE, indicating comparatively lower error levels.
Similarly, MAP reflected the second-lowest MAE, underscoring the models' proficiency in handling diverse BP parameters.
A crucial finding indicates that across all models, DBP can be predicted with greater precision than SBP\@.
However, this finding also signifies a disappointment, given that SBP serves as a critical metric in hypertension detection,
a prevalent cause of mortality globally according to the World Health Statistics~\cite{WorldHealthStatistics2023}.

The study yielded an intriguing observation regarding the impact of feature weight adjustment on model performance, as delineated in Table~\ref{tab:test_validate_rmse_mae}.
An illustration of this phenomenon is exemplified by the Bi-GRU model, which exhibited a test MAE of 7.844, and displayed a notable increase, reaching 9.811 after weight adjustment.
Conversely, the validation MAE following feature weight adjustment decreased, dropping from 13.145 to 11.614.
This juxtaposition suggests a mixed influence of feature weight adjustments on the models' capacity for generalization.
While certain models experienced a deterioration in testing performance, indicated by the elevated MAE, the enhanced validation performance,
crucial for assessing the models' ability to generalize to dissimilar datasets, was conspicuous.
The implication is that the feature weight adjustments potentially aided in mitigating overfitting, thereby enabling the models to more effectively discern underlying data patterns during validation.
However, this finding also hints at a potential drawback, whereby these models might exhibit suboptimal performance when presented with new, unseen data from the same distribution.
This underscores the significance of careful model selection and tuning to achieve the desired balance between training and validation performance.
A more in-depth exploration into the precise impact of feature weight adjustments on model generalization would be advantageous in refining the models' predictive efficacy.

% Strengths
The strength of this study lies in its robustness against overfitting despite the challenging task of predicting BP from a vast amount of data.
Despite the models showing suboptimal performance on the extensive datasets, the absence of overfitting indicates that the models were not merely memorizing the training data but were instead learning underlying patterns.
This is crucial for ensuring the generalizability of the models to new, unseen data, a key factor in their utility in real-world applications.

Moreover, the simulation of real-world data through the use of MIMIC-IV, a distinct and diverse data source, enhances the study's strength.
By incorporating data from a different source, the models were exposed to varied patterns and distributions, which is essential for creating a universal ML model.
This approach helps in capturing the diverse range of scenarios and conditions that can be encountered in clinical settings, making the models more adaptable and reliable when deployed in practice.
