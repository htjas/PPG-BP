    Cardiovascular diseases rank as the foremost cause of global mortality, with hypertension as a significant contributing factor.
    The need for continuous blood pressure (BP) monitoring is paramount, especially in resource-constrained settings.
    This study explores utilizing Photoplethysmography (PPG) technology for BP estimation in Intensive Care Unit (ICU) patients, delving into signal pre-processing techniques,
    PPG feature extraction, and machine learning (ML) algorithms for precise BP monitoring.

    \vspace{0.4cm}

    Drawing upon ICU data from Multiparameter Intelligent Monitoring in Intensive Care (MIMIC) databases, comprising 22,083 10-minute segments from MIMIC-III and 5,508 identical length segments from MIMIC-IV,
    rigorous signal processing and feature extraction procedures yielded a total of 1,918,623 and 110,408 median data points from MIMIC-III and MIMIC-IV respectively.
    Each data point encompassed 34 independent (input) PPG features and 3 target (output) arterial blood pressure (ABP) features in synchrony.
    The MIMIC-III dataset was utilized for model training and testing, while the MIMIC-IV dataset served for validation purposes.
    Within the scope of this study, seven distinct ML models were developed and assessed.

    \vspace{0.4cm}

    The Multi-Layer Perceptron achieved the best overall ABP prediction accuracy, with a mean absolute error (MAE) and standard deviation (STD) of 11.072 ± 9.725 mmHg.
    Notably, the highest accuracy in predicting Diastolic Blood Pressure (DBP) was observed, with an MAE and STD of 8.328 ± 5.814 mmHg.
    Key features such as Diastolic Time, Resistive Index, and Normalized Power at Peak exhibited significant correlation and impactful contributions to the models' predictive capabilities.
    A crucial finding indicates that across all models, DBP can be predicted with greater precision than systolic BP\@.

    \vspace{0.4cm}

    Despite not meeting international standards for acceptable accuracy in BP measurement, the recorded metrics offer valuable insights into cardiovascular dynamics captured through PPG technology.
    Developing a universal BP prediction model solely from PPG technology poses a significant challenge, emphasizing the personalized nature of PPG waveforms
    and the impact of individual physiological characteristics on BP predictions.
    Future directions include implementing calibration techniques, utilizing verifiable patient data for model training, and deriving subject-specific parameters
    for advancing nuanced and effective BP prediction models tailored to individual patients.