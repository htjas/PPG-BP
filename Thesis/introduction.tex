\subsection{Subject and Motivation}
\label{subsec:subject_motivation}

Cardiovascular diseases (CVDs) are the leading cause of death worldwide, according to WHO publishing statistics~\cite{organizationWorldHealthStatistics2023}.
One of the main factors contributing to CVDs is Hypertension.
It is the leading risk factor for mortality, and is ranked third as a cause of disability-adjusted life-years~\cite{ezzatiSelectedMajorRisk2002}.
Currently, there is a significant need for continuous blood pressure (BP) monitoring due to various factors.
Primarily, while hypertension is a manageable condition, the availability of accurate high BP detection remains scarce, especially in low-resource environments \cite{burtPrevalenceHypertensionUS1995}.
Additionally, blood pressure (BP) is subject to rapid fluctuations influenced by various factors, including stress, emotions, dietary intake, physical activity, and medication usage~\cite{poonCufflessNoninvasiveMeasurements2005}.
Continuous monitoring of blood pressure, rather than relying on isolated measurements, plays a vital role in the early detection and treatment of hypertension~\cite{el-hajjDeepLearningModels2021}.

The current accurate methods for measuring BP continuously are either invasive or involving a cuff-mechanism.
Catheterization is internationally recognized as the \enquote{gold standard} for obtaining the most accurate measurement of continuous blood pressure~\cite{sharmaCuffLessContinuousBlood2017}.
However, due to its invasive nature and limited applicability to hospital settings, this method requires medical intervention, which renders it inconvenient for everyday use.

While cuff-based devices are commonly utilized for this objective, it is worth noting that over 30\% of home blood pressure monitors are found to be inaccurate, rendering continuous measurement unfeasible~\cite{leungHypertensionCanada20162016}.
Moreover, this approach relies on the individual consciously and intentionally engaging in manual blood pressure monitoring, which poses limitations and might be often overlooked.

An ideal technology for measuring blood pressure should have the following attributes: non-invasiveness, cuffless operation, optical functionality, wearable design, and cost-effectiveness~\cite{el-hajjDeepLearningModels2021}.
One approach satisfying these requirements is the estimation of BP from a single measurement PPG sensor.
This approach, using two modes, reflectance and transmission, has gained an increasing amount of attention in the literature due its simplicity, and ability to provide continuous and cuffless measurement~\cite{el-hajjDeepLearningModels2021}.
Typically, the photoplethysmography (PPG) technique has been traditionally employed in healthcare settings to measure heart rate~\cite{reyesWirelessPhotoplethysmographicDevice2012} and blood oxygen saturation using a pulse oximeter~\cite{yoonMultipleDiagnosisBased2002}.

Nevertheless, establishing a straightforward, distinct, and continuous relationship between these characteristics and blood pressure (BP) has proven to be challenging.
To address this, the approach heavily depends on signal pre-processing techniques, extracting PPG features, and utilizing machine learning algorithms to estimate BP based on these features~\cite{el-hajjDeepLearningModels2021}.
A recent scoping review by Knight et al.\ concluded that PPG can be successfully used to continuously measure BP, by evaluating latest publications and finding over 80\% accuracy in detecting hypertension~\cite{knightAccuracyWearablePhotoplethysmography2022}.

This study examines the current methods and aims to develop efficient approaches for the continuous and accurate measurement of blood pressure using PPG and addresses the following research questions: 

\begin{enumerate}
\item \textbf{What is the relationship between PPG data and blood pressure among ICU patients?}

\item \textbf{Can PPG-based data be used to estimate blood pressure accurately?}

\end{enumerate}

\subsection{Tasks and Objectives}
\label{subsec:tasks_objectives}

The tasks of the thesis are as follows:

\begin{enumerate}
\item Signal Processing: to find an optimal data fetching and filtering approach from available MIMIC Databases.
\item Likewise, to create a consistent algorithm for key feature extraction.
\item Machine Learning: to develop a model based on the resulting features from Signal Processing, to reliably predict BP from PPG\@.
\end{enumerate}

\subsection{Structure of the Thesis}
\label{subsec:structure}

This thesis is organized as follows:

In Chapter~\ref{sec:background}, the foundations of the used terms and prerequisites for the methods are explained.
For example, in~\ref{subsec:med_background}, the terms \enquote{Blood Pressure} (\ref{subsubsec:bp}) and \enquote{Photoplethysmography} (\ref{subsubsec:ppg}) are discussed.
Furthermore, the general structure of MIMIC databases is explained (\ref{subsubsec:mimic3},~\ref{subsubsec:mimic4}). In addition, the essential information about the Computing part of this research is provided (\ref{subsec:computing_background}).

In Chapter~\ref{sec:methods}, the methodology is explained.

Chapter~\ref{sec:results} presents the results.

The focus of Chapters~\ref{sec:discussion} and~\ref{sec:conclusion} is on summarizing the work.
Here, both the future prospects for this research field and the next steps in relation to broader scope projects are presented.