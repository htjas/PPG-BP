\subsection{Subject and Motivation}
\label{subsec:subject_motivation}

\ac{CVDs} are the leading cause of death worldwide, according to WHO publishing statistics~\cite{WorldHealthStatistics2023}.
One of the main factors contributing to CVDs is Hypertension.
It is the leading risk factor for mortality, and is ranked third as a cause of disability-adjusted life-years~\cite{ezzatiSelectedMajorRisk2002}.
Currently, there is a significant need for continuous \ac{BP} monitoring due to various factors.
Primarily, while hypertension is a manageable condition, the availability of accurate high BP detection remains scarce, especially in low-resource environments~\cite{burtPrevalenceHypertensionUS1995}.
Additionally, BP is subject to rapid fluctuations influenced by various factors, including stress, emotions, dietary intake, physical activity, and medication usage~\cite{poonCufflessNoninvasiveMeasurements2005}.
Continuous monitoring of BP, rather than relying on isolated measurements, plays a vital role in the early detection and treatment of hypertension~\cite{el-hajjDeepLearningModels2021}.

The current accurate methods for measuring BP continuously are either invasive or involving a cuff-mechanism.
Catheterization is internationally recognized as the \enquote{gold standard} for obtaining the most accurate measurement of continuous BP~\cite{sharmaCuffLessContinuousBlood2017}.
However, due to its invasive nature and limited applicability to hospital settings, this method requires medical intervention, which renders it inconvenient for everyday use.

While cuff-based devices are commonly utilized for this objective, it is worth noting that around 30\% of home BP monitors are found to be inaccurate, rendering continuous measurement unfeasible~\cite{leungHypertensionCanada20162016, seboBloodPressureMeasurements2014}.
Moreover, this approach relies on the individual consciously and intentionally engaging in manual BP monitoring, which poses limitations and might be often overlooked.

An ideal technology for measuring BP should have the following attributes: non-invasiveness, cuffless operation, optical functionality, wearable design, and cost-effect\-iveness~\cite{el-hajjDeepLearningModels2021}.
One approach satisfying these requirements is the estimation of BP from a single measurement \ac{PPG} sensor.
This approach, using two modes, reflectance and transmission, has gained an increasing amount of attention in the literature due to its straightforward nature, and capacity to offer continuous and cuffless measurement~\cite{el-hajjDeepLearningModels2021}.
Traditionally, the PPG technique has been employed in healthcare settings to measure heart rate~\cite{reyesWirelessPhotoplethysmographicDevice2012} and blood oxygen saturation using a pulse oximeter~\cite{yoonMultipleDiagnosisBased2002}.

Nevertheless, establishing a straightforward, distinct, and continuous relationship between these characteristics and BP has proven to be challenging.
To address this, the approach heavily depends on signal pre-processing techniques, extracting PPG features, and utilizing \ac{ML} algorithms to estimate BP based on these features~\cite{el-hajjDeepLearningModels2021}.
A recent scoping review by Knight et al.\ concluded that PPG can be successfully used to continuously measure BP, by evaluating latest publications and finding over 80\% accuracy in detecting hypertension~\cite{knightAccuracyWearablePhotoplethysmography2022}.

The \ac{ICU} is a medical domain where the significance of non-invasive BP measurement methods is profound.
An expeditious and painless methodology holds benefits for both patients and medical practitioners.
Additionally, PPG technology is presently employed in ICUs through oximeters~\cite{aoyagiPulseOximetryIts2002}, primarily for monitoring \ac{SpO2} and pulse rates.
As a result, the incorporation of these techniques would not necessitate substantial implementation resources

\vspace{0.2cm}

This investigation delves into existing methodologies and seeks to devise effective strategies for the continual and precise monitoring of BP via PPG\@.
The study centers on addressing the following research question:

\begin{itemize}
    \item \textbf{Can the PPG technology be used to estimate BP of ICU patients?}
\end{itemize}

\subsection{Tasks and Objectives}
\label{subsec:tasks_objectives}

The tasks of the thesis are as follows:

\begin{enumerate}
    \item To find an optimal data fetching and filtering approach from available ICU data sources.
    \item To create a consistent algorithm for key PPG feature extraction.
    \item To develop and test various ML models based on the extracted features, for reliable prediction of BP from PPG\@.
\end{enumerate}

\subsection{Structure of the Thesis}
\label{subsec:structure}

This thesis is organized as follows:

In Chapter~\ref{sec:background}, the fundamental concepts and prerequisites essential for the methods are delineated.
This includes the elaboration of medical theory regarding Blood Pressure~(\ref{subsubsec:bp}) and Photoplethysmography~(\ref{subsubsec:ppg}) in~\ref{subsec:med_background}.
Additionally, the structure of the ICU data sources utilized in this study is detailed (\ref{subsubsec:mimic}).
The computational aspects of this research, such as state-of-the-art signal processing (\ref{subsubsec:signal_processing}) and machine learning (\ref{subsubsec:machine_learning}) techniques, are also expounded upon (\ref{subsec:computing_background}).

Chapter~\ref{sec:methods} provides an exposition of the methodology employed.
It covers the processes of Data Fetching (\ref{subsec:data-fetching}), Signal Processing (\ref{subsec:sp_methods}), and Machine Learning (\ref{subsec:ml_methods}).

The results are disclosed in Chapter~\ref{sec:results}.
This section encompasses the total samples retrieved and the features extracted (\ref{subsec:results_data}), as well as the final evaluation metrics of the ML models (\ref{subsec:machine_learning}).

Chapters~\ref{sec:discussion} and~\ref{sec:conclusion} are dedicated to the discussion and conclusion of the study.
Here, the findings, limitations, and future directions for this research field are delineated, along with potential avenues for broader-scale projects and subsequent steps.