\subsection{Filters}\label{subsec:code_filters}

\begin{lstlisting}[language=Python,label={lst:filters.py}, basicstyle=\scriptsize]

from scipy.ndimage import gaussian_filter1d
import scipy as sp
import matplotlib.pyplot as plt
import numpy as np


def pre_process_data(abp, ppg, fs):
    # Gaussian filter
    g_abp = gaussian_filter1d(abp, sigma=2)
    g_ppg = gaussian_filter1d(ppg, sigma=2)

    # Gaussian Median filter
    med_a = np.median(abp)
    std_a = np.std(abp)
    sigma_a = med_a / std_a * 0.5
    gm_abp = gaussian_filter1d(abp, sigma=sigma_a)
    med_p = np.median(ppg)
    std_p = np.std(ppg)
    sigma_p = med_p / std_p * 0.3
    gm_ppg = gaussian_filter1d(ppg, sigma=sigma_p)

    # Savitzky-Golay filter
    sg_abp = filter_savgol(abp)
    sg_ppg = filter_savgol(ppg)

    # Butterworth Lowpass filter
    bl_abp = butter_lowpass_filter(abp, 5, fs, 4)
    bl_ppg = butter_lowpass_filter(ppg, 5, fs, 4)

    # Butterworth filter
    b_abp = filter_butterworth(abp, fs)
    b_ppg = filter_butterworth(ppg, fs)

    # Chebyshev Lowpass filter
    cl_abp = chebyshev2_lowpass_filter(abp, 2, fs)
    cl_ppg = chebyshev2_lowpass_filter(ppg, 2, fs)

    # Chebyshev filter
    c_abp = filter_chebyshev(abp, fs)
    c_ppg = filter_chebyshev(ppg, fs)

    # Whiskers filter
    w_abp = whiskers_filter(abp)
    w_ppg = whiskers_filter(ppg)


def filter_savgol(x):
    x_f = sp.savgol_filter(x, 51, 4)
    return x_f


def butter_lowpass_filter(data, cutoff_frequency, fs, order):
    nyquist = 0.5 * fs
    normal_cutoff = cutoff_frequency / nyquist
    b, a = sp.butter(order, normal_cutoff, btype='low', analog=False, output='ba')

    filtered_data = sp.lfilter(b, a, data)
    return filtered_data


def chebyshev2_lowpass_filter(data, cutoff_freq, fs):
    nyquist = 0.5 * fs
    normal_cutoff = cutoff_freq / nyquist
    b, a = sp.cheby2(4, 2, normal_cutoff, btype='low', analog=False)
    y = sp.lfilter(b, a, data)
    return y


def filter_butterworth(data, fs):
    lpf_cutoff = 1.25  # Hz
    hpf_cutoff = 31  # Hz
    sos_butt = sp.butter(10,
                         [lpf_cutoff, hpf_cutoff],
                         btype='bp',
                         analog=False,
                         output='sos',
                         fs=fs)
    w, h = sp.sosfreqz(sos_butt,
                       2000,
                       fs=fs)
    sos = sos_butt, w, h
    return sp.sosfiltfilt(sos[0], data)


def filter_chebyshev(data, fs):
    lpf_cutoff = 1.2  # Hz
    hpf_cutoff = 60  # Hz
    sos_cheb = sp.cheby2(4,
                         20,
                         [lpf_cutoff, hpf_cutoff],
                         btype='bp',
                         analog=False,
                         output='sos',
                         fs=fs)
    w, h = sp.sosfreqz(sos_cheb,
                       2000,
                       fs=fs)
    sos = sos_cheb, w, h
    return sp.sosfiltfilt(sos[0], data)


def whiskers_filter(data):
    bp = plt.boxplot(data)
    plt.close()

    # get lower and upper amplitude thresholds
    whiskers = [whiskers.get_ydata() for whiskers in bp['whiskers']]
    lower_amp = whiskers[0][1]
    upper_amp = whiskers[1][1]

    # get all indexes of outliers, outside the amplitude thresholds
    ind_outliers = []
    for i in range(len(data)):
        if data[i] < lower_amp or data[i] > upper_amp:
            ind_outliers.append(i)

    if len(ind_outliers) != 0:
        # get all grouped arrays with values not within the whiskers range
        ind_consecutives = []
        current_group = [ind_outliers[0]]
        for i in range(1, len(ind_outliers)):
            if ind_outliers[i] == ind_outliers[i - 1] + 1:
                current_group.append(ind_outliers[i])
            else:
                ind_consecutives.append(current_group)
                current_group = [ind_outliers[i]]
        ind_consecutives.append(current_group)

        for array_indexes in ind_consecutives:
            array_values = data[array_indexes]
            # get top (min or max) value of each consecutive group
            if array_values[0] < lower_amp:
                top_val = min(array_values)
                top_ind = array_indexes[np.argmin(array_values)]
                coef = top_val / lower_amp
            else:
                top_val = max(array_values)
                top_ind = array_indexes[np.argmax(array_values)]
                coef = top_val / upper_amp
            # calculate and (not) assign the top value according to attenuating coefficient
            adj_top_val = top_val / coef * 1.01
            if lower_amp > top_val > adj_top_val:
                continue
            if upper_amp < top_val < adj_top_val:
                continue
            data[top_ind] = adj_top_val
            # check if first group index is also first overall index
            if array_indexes[0] == 0:
                one_minus_threshold_val = data[array_indexes[0]]
                one_minus_threshold_ind = array_indexes[0]
            else:
                one_minus_threshold_val = data[array_indexes[0] - 1]
                one_minus_threshold_ind = array_indexes[0] - 1
            # check if last group index is also last overall index
            if array_indexes[-1] == len(data) - 1:
                one_plus_threshold_val = data[array_indexes[-1]]
                one_plus_threshold_ind = array_indexes[-1]
            else:
                one_plus_threshold_val = data[array_indexes[-1] + 1]
                one_plus_threshold_ind = array_indexes[-1] + 1
            # create function for calculation of other new values
            distance_to_top = top_ind - one_minus_threshold_ind
            if distance_to_top == 0:
                distance_to_top = 1
            distance_to_bottom = one_plus_threshold_ind - top_ind
            if distance_to_bottom == 0:
                distance_to_bottom = 1
            f1 = (adj_top_val - one_minus_threshold_val) / distance_to_top
            f2 = (one_plus_threshold_val - adj_top_val) / distance_to_bottom
            # assign new values
            x1, x2 = 1, 1
            for ind in array_indexes:
                if ind < top_ind:
                    data[ind] = (f1 * x1) + one_minus_threshold_val
                    x1 += 1
                elif ind > top_ind:
                    data[ind] = (f2 * x2) + adj_top_val
                    x2 += 1

    return data

\end{lstlisting}

\newpage

\subsection{Beats}
\label{subsec:code_beats}

\begin{lstlisting}[language=Python,label={lst:beats.py}, basicstyle=\scriptsize]
import scipy as sp
import numpy as np


def get_optimal_beats_lists(abp, ppg, fs):
    """
    Signal processing script for examining
    which default beat detection algorithm is the most efficient
    :param abp: List of ABP data
    :param ppg: List of PPG data
    :param fs: Sampling frequency
    :return: Lists of optimally detected ABP and PPG pulses
    """
    abp_beats1 = pulse_detection(abp, 'd2max', len(abp) / fs, 'abp')
    ppg_beats1 = pulse_detection(ppg, 'd2max', len(ppg) / fs, 'ppg')
    abp_beats2 = pulse_detection(abp, 'upslopes', len(abp) / fs, 'abp')
    ppg_beats2 = pulse_detection(ppg, 'upslopes', len(ppg) / fs, 'ppg')
    abp_beats3 = pulse_detection(abp, 'delineator', len(abp) / fs, 'abp')
    ppg_beats3 = pulse_detection(ppg, 'delineator', len(ppg) / fs, 'ppg')

    avg1 = (len(abp_beats1) + len(ppg_beats1)) / 2
    avg2 = (len(abp_beats2) + len(ppg_beats2)) / 2
    avg3 = (len(abp_beats3) + len(ppg_beats3)) / 2

    sorted_avg = sorted([avg1, avg2, avg3])
    diff_avg12 = sorted_avg[1] - sorted_avg[0]
    diff_avg23 = sorted_avg[2] - sorted_avg[1]

    outlier = None
    if diff_avg23 > diff_avg12 * 2:
        outlier = sorted_avg[2]
    elif diff_avg12 > diff_avg23 * 2:
        outlier = sorted_avg[0]

    diff1 = abs(len(abp_beats1) - len(ppg_beats1))
    diff2 = abs(len(abp_beats2) - len(ppg_beats2))
    diff3 = abs(len(abp_beats3) - len(ppg_beats3))

    diffper1 = diff1 / avg1 * 100
    diffper2 = diff2 / avg2 * 100
    diffper3 = diff3 / avg3 * 100

    if outlier is None:
        abp_opt, ppg_opt = get_best_beats_from_diffper(diffper1, diffper2, diffper3,
                                                       abp_beats1, abp_beats2, abp_beats3,
                                                       ppg_beats1, ppg_beats2, ppg_beats3)
    else:
        if outlier == avg1:
            diffper1 = 100
        elif outlier == avg2:
            diffper2 = 100
        elif outlier == avg3:
            diffper3 = 100
        abp_opt, ppg_opt = get_best_beats_from_diffper(diffper1, diffper2, diffper3,
                                                   abp_beats1, abp_beats2, abp_beats3,
                                                   ppg_beats1, ppg_beats2, ppg_beats3)
    return abp_opt, ppg_opt


def pulse_detection(data, algorithm, duration):
    temp_fs = 125
    beats = pulse_detect(data, temp_fs, 5, algorithm, duration)
    return beats


def get_best_beats_from_diffper(diffper1, diffper2, diffper3,
                                abp_beats1, abp_beats2, abp_beats3,
                                ppg_beats1, ppg_beats2, ppg_beats3):
    smallest_diffper = min(diffper1, diffper2, diffper3)

    abp_opt = []
    ppg_opt = []

    if smallest_diffper == diffper1:
        abp_opt = abp_beats1
        ppg_opt = ppg_beats1
    elif smallest_diffper == diffper2:
        abp_opt = abp_beats2
        ppg_opt = ppg_beats2
    elif smallest_diffper == diffper3:
        abp_opt = abp_beats3
        ppg_opt = ppg_beats3

    return abp_opt, ppg_opt

def pulse_detect(x, fs, w, alg, dur):
    """
    Description: Pulse detection and correction from pulsatile signals
    Developed by: Elisa Mejia-Mejia
                   City, University of London
    Full code available at:
    https://wfdb.io/mimic_wfdb_tutorials/tutorial/notebooks/beat-detection.html
    """


def get_beats_from_mean_crossings(abp, ppg):
    mean_value = np.mean(abp)
    above_mean = abp > mean_value
    count_a = np.count_nonzero(np.diff(above_mean.astype(int)) == -1)
    mean_value = np.mean(ppg)
    above_mean = ppg > mean_value
    count_p = np.count_nonzero(np.diff(above_mean.astype(int)) == -1)
    abp_beat_interval = len(abp) / ((count_a + count_p) / 2)
    ppg_beat_interval = abp_beat_interval

    abp_beats, _ = sp.find_peaks(abp, distance=abp_beat_interval, prominence=5)
    ppg_beats, _ = sp.find_peaks(ppg, distance=ppg_beat_interval, prominence=0.01)

    return abp_beats, ppg_beats

\end{lstlisting}

\newpage

\subsection{Signal Processing}
\label{subsec:code_sp}

\begin{lstlisting}[language=Python,label={lst:sp.py}, basicstyle=\scriptsize]
import numpy as np
import scipy as sp


def signal_processing(seg_name, abp, ppg, fs):
    # First iteration of beat finding (MIMIC default methods)
    abp_beats, ppg_beats = get_optimal_beats_lists(abp, ppg, seg_name)

    # For comparison: beat detection from mean crossing
    beats_a, beats_p = get_beats_from_mean_crossings(abp, ppg)
    is_larger = len(beats_a) + len(beats_p) > len(abp_beats) + len(beats_p)
    is_closer = abs(len(beats_a) - len(beats_p)) < abs(len(abp_beats) - len(ppg_beats))
    if is_larger and is_closer:
        abp_beats = beats_a
        ppg_beats = beats_p

    # Second iteration: Peak finding (SP manual methods)
    abp_beat_interval = len(abp) / len(abp_beats)
    ppg_beat_interval = len(ppg) / len(ppg_beats)
    abp_beats, _ = sp.find_peaks(abp, distance=abp_beat_interval * .75, prominence=0.5)
    ppg_beats, _ = sp.find_peaks(ppg, distance=ppg_beat_interval * .75, prominence=0.01)
    if len(abp_beats) < 100 or len(ppg_beats) < 100:
        raise Exception('Signal Processing failed: a substantial amount of beats not found')

    # Signal synchronization : delay approx = 18 (288 ms)
    abp, ppg = synchronization(abp, ppg, abp_beats, ppg_beats)

    # Third iteration: Peak finding
    abp_beats, _ = sp.find_peaks(abp, distance=abp_beat_interval * .75, prominence=0.5)
    ppg_beats, _ = sp.find_peaks(ppg, distance=ppg_beat_interval * .75, prominence=0.01)
    normal_length_a, normal_length_p = len(abp_beats), len(ppg_beats)

    # Beat grouping
    abp_beats, ppg_beats = group_beats(abp_beats, ppg_beats)

    if max(normal_length_a, normal_length_p) > max(len(abp_beats), len(ppg_beats)) * 1.05:
        raise Exception(f"too big of a difference after grouping -"
                        f"{max(normal_length_a,normal_length_p)-max(len(abp_beats),len(ppg_beats))}")

    abp_hr = round(len(abp_beats) / (len(abp) / fs) * 60, 1)
    ppg_hr = round(len(ppg_beats) / (len(ppg) / fs) * 60, 1)
    print(f"\tABP beats - {len(abp_beats)}, Heart Rate - {abp_hr}")
    print(f"\tPPG beats - {len(ppg_beats)}, Heart Rate - {ppg_hr}")

    return {
        'abp': abp,
        'ppg': ppg,
        'abp_beats': abp_beats,
        'ppg_beats': ppg_beats,
        'abp_hr': abp_hr,
        'ppg_hr': ppg_hr
    }


def synchronization(abp, ppg, abp_beats, ppg_beats):
    # Find closest PPG beat to first ABP beat
    first_ppg_ind, first_abp = None, None
    for i in range(0, len(abp_beats)):
        first_abp = abp_beats[i]
        differences = ppg_beats - first_abp
        arr = [num for num in differences if num > 0]
        if len(arr) == 0:
            raise Exception("Beat synchronisation failed: no proximal beats found")
        smallest_diff = min(arr)
        if smallest_diff < 50:
            first_ppg_ind, = np.argwhere(differences == smallest_diff)[0]
            break
    first_ppg = ppg_beats[first_ppg_ind]

    # Find closest ABP beat to last PPG beat
    last_ppg_ind = len(ppg_beats) - 1
    last_ppg = ppg_beats[last_ppg_ind]
    differences = abs(abp_beats - last_ppg)
    last_abp_ind = np.argmin(differences)
    last_abp = abp_beats[last_abp_ind]

    # Splice original ABP and PPG arrays
    new_abp = abp[first_abp:last_abp]
    new_ppg = ppg[first_ppg:last_ppg]

    return new_abp, new_ppg


def group_beats(abp_beats, ppg_beats):
    i = 0
    while i < min(len(abp_beats), len(ppg_beats)):
        a = abp_beats[i]
        p = ppg_beats[i]
        if p - 40 <= a <= p + 40:
            i += 1
        else:
            if a < p:
                abp_beats = abp_beats[abp_beats != a]
            elif a > p:
                ppg_beats = ppg_beats[ppg_beats != p]
    abp_beats, ppg_beats = equal_out_by_shortening(abp_beats, ppg_beats)
    return abp_beats, ppg_beats


def equal_out_by_shortening(a_ts, p_ts):
    if len(a_ts) > len(p_ts):
        diff = len(a_ts) - len(p_ts)
        a_ts = a_ts[:-diff]
    elif len(a_ts) < len(p_ts):
        diff = len(p_ts) - len(a_ts)
        p_ts = p_ts[:-diff]
    return a_ts, p_ts
\end{lstlisting}

\newpage

\subsection{Fiducial Points}
\label{subsec:code_fidp}

\begin{lstlisting}[language=Python,label={lst:fidp.py}, basicstyle=\scriptsize]
import numpy as np
import scipy as sp


def fiducial_points(x, pks, fs, vis, header):
    """
    Description: Pulse detection and correction from pulsatile signals
    Inputs:  x, array with pulsatile signal [user defined units]
             pks, array with the position of the peaks [number of samples]
             fs, sampling rate of signal [Hz]
             vis, visualisation option [True, False]
    Outputs: fidp, dictionary with the positions of several fiducial points
             for the cardiac cycles [number of samples]

    Fiducial points:  1: Systolic peak (pks)
                      2: Onset, as the minimum before the systolic peak (ons)
                      3: Onset, using the tangent intersection method (ti)
                      4: Diastolic peak (dpk)
                      5: Maximum slope (m1d)
                      6: a point from second derivative PPG (a2d)
                      7: b point from second derivative PPG (b2d)
                      8: c point from second derivative PPG (c2d)
                      9: d point from second derivative PPG (d2d)
                      10: e point from second derivative PPG (e2d)
                      11: p1 from the third derivative PPG (p1)
                      12: p2 from the third derivative PPG (p2)

    Libraries: NumPy (as np), SciPy (Signal, as sp), Matplotlib (PyPlot, as plt)

    Version: 1.0 - June 2022

    Developed by: Elisa Mejia-Mejia
                   City, University of London

    Edited by: Peter Charlton (see "Added by PC"), Hugas Jasinskas (see "Adjusted by HJ")

    """
    # First, second and third derivatives
    d1x = sp.savgol_filter(x, 9, 5, deriv=1)
    d2x = sp.savgol_filter(x, 9, 5, deriv=2)
    d3x = sp.savgol_filter(x, 9, 5, deriv=3)

    # Search in time series: Onsets between consecutive peaks
    ons = np.empty(0)
    for i in range(len(pks) - 1):
        start = pks[i]
        stop = pks[i + 1]
        ibi = x[start:stop]
        aux_ons, = np.where(ibi == np.min(ibi))
        if len(aux_ons) > 1:
            aux_ons = aux_ons[0]
        ind_ons = aux_ons.astype(int)
        ons = np.append(ons, ind_ons + start)
    ons = ons.astype(int)

    # Search in time series: Diastolic peak and dicrotic notch between consecutive onsets
    dia = np.empty(0)
    dic = np.empty(0)
    for i in range(len(ons) - 1):
        start = ons[i]
        stop = ons[i + 1]
        ind_pks, = np.intersect1d(np.where(pks < stop), np.where(pks > start))
        ind_pks = pks[ind_pks]
        ibi_portion = x[ind_pks:stop]
        ibi_2d_portion = d2x[ind_pks:stop]
        aux_dic, _ = sp.find_peaks(ibi_2d_portion)
        aux_dic = aux_dic.astype(int)
        aux_dia, _ = sp.find_peaks(-ibi_2d_portion)
        aux_dia = aux_dia.astype(int)
        if len(aux_dic) != 0:
            ind_max, = np.where(ibi_2d_portion[aux_dic] == np.max(ibi_2d_portion[aux_dic]))
            aux_dic_max = aux_dic[ind_max][0]  # Adjusted by HJ
            if len(aux_dia) != 0:
                nearest = aux_dia - aux_dic_max
                aux_dic = aux_dic_max
                dic = np.append(dic, (aux_dic + ind_pks).astype(int))
                ind_dia, = np.where(nearest > 0)
                aux_dia = aux_dia[ind_dia]
                nearest = nearest[ind_dia]
                if len(nearest) != 0:
                    ind_nearest, = np.where(nearest == np.min(nearest))
                    aux_dia = aux_dia[ind_nearest]
                    dia = np.append(dia, (aux_dia + ind_pks).astype(int))
            else:
                dic = np.append(dic, (aux_dic_max + ind_pks).astype(int))
    dia = dia.astype(int)
    dic = dic.astype(int)

    # Search in D1: Maximum slope point
    m1d = np.empty(0)
    for i in range(len(ons) - 1):
        start = ons[i]
        stop = ons[i + 1]
        ind_pks, = np.intersect1d(np.where(pks < stop), np.where(pks > start))
        ind_pks = pks[ind_pks]
        ibi_portion = x[start:ind_pks]
        ibi_1d_portion = d1x[start:ind_pks]
        aux_m1d, _ = sp.find_peaks(ibi_1d_portion)
        aux_m1d = aux_m1d.astype(int)
        if len(aux_m1d) != 0:
            ind_max, = np.where(ibi_1d_portion[aux_m1d] == np.max(ibi_1d_portion[aux_m1d]))
            aux_m1d_max = aux_m1d[ind_max]
            if len(aux_m1d_max) > 1:
                aux_m1d_max = aux_m1d_max[0]
            m1d = np.append(m1d, (aux_m1d_max + start).astype(int))
    m1d = m1d.astype(int)

    # Search in time series: Tangent intersection points
    tip = np.empty(0)
    for i in range(len(ons) - 1):
        start = ons[i]
        stop = ons[i + 1]
        ibi_portion = x[start:stop]
        ibi_1d_portion = d1x[start:stop]
        low_stop = np.where(m1d < stop)  # Adjusted by HJ from here
        high_start = np.where(m1d > start)
        if np.intersect1d(low_stop, high_start).size == 0:
            continue
        ind_m1d, = np.intersect1d(low_stop, high_start)  # to here
        ind_m1d = m1d[ind_m1d] - start
        aux_tip = np.round(((ibi_portion[0] - ibi_portion[ind_m1d]) /
                             ibi_1d_portion[ind_m1d]) + ind_m1d)
        aux_tip = aux_tip.astype(int)
        tip = np.append(tip, (aux_tip + start).astype(int))
    tip = tip.astype(int)

    # Search in D2: A, B, C, D and E points
    a2d = np.empty(0)
    b2d = np.empty(0)
    c2d = np.empty(0)
    d2d = np.empty(0)
    e2d = np.empty(0)
    for i in range(len(ons) - 1):
        start = ons[i]
        stop = ons[i + 1]
        ibi_portion = x[start:stop]
        ibi_1d_portion = d1x[start:stop]
        ibi_2d_portion = d2x[start:stop]
        ind_m1d = np.intersect1d(np.where(m1d > start), np.where(m1d < stop))
        ind_m1d = m1d[ind_m1d]
        aux_m2d_pks, _ = sp.find_peaks(ibi_2d_portion)
        aux_m2d_ons, _ = sp.find_peaks(-ibi_2d_portion)
        if len(aux_m2d_pks) == 0 or len(aux_m2d_ons) == 0:  # Augmented by HJ
            continue
        # a point:
        ind_a, = np.where(ibi_2d_portion[aux_m2d_pks] == np.max(ibi_2d_portion[aux_m2d_pks]))
        ind_a = aux_m2d_pks[ind_a]
        ind_a = ind_a[0]  # Adjusted by HJ
        if (ind_a < ind_m1d):
            a2d = np.append(a2d, ind_a + start)
            # b point:
            ind_b = np.where(ibi_2d_portion[aux_m2d_ons] == np.min(ibi_2d_portion[aux_m2d_ons]))
            ind_b = aux_m2d_ons[ind_b]
            ind_b = ind_b[0]  # Adjusted by HJ
            if (ind_b > ind_a) and (ind_b < len(ibi_2d_portion)):
                b2d = np.append(b2d, ind_b + start)
        # e point:
        if len(ind_m1d) == 0: # Augmented by HJ
            continue
        ind_e, = np.where(aux_m2d_pks > ind_m1d - start)
        aux_m2d_pks = aux_m2d_pks[ind_e]
        ind_e, = np.where(aux_m2d_pks < 0.6 * len(ibi_2d_portion))
        ind_e = aux_m2d_pks[ind_e]
        if len(ind_e) >= 1:
            if len(ind_e) >= 2:
                ind_e = ind_e[1]
            e2d = np.append(e2d, ind_e + start)
            # c point:
            ind_c, = np.where(aux_m2d_pks < ind_e)
            if len(ind_c) != 0:
                ind_c_aux = aux_m2d_pks[ind_c]
                ind_c, = np.where(ibi_2d_portion[ind_c_aux] == np.max(ibi_2d_portion[ind_c_aux]))
                ind_c = ind_c_aux[ind_c]
                if len(ind_c) != 0:
                    c2d = np.append(c2d, ind_c + start)
            else:
                aux_m1d_ons, _ = sp.find_peaks(-ibi_1d_portion)
                ind_c, = np.where(aux_m1d_ons < ind_e)
                ind_c_aux = aux_m1d_ons[ind_c]
                if len(ind_c) != 0:
                    ind_c, = np.where(ind_c_aux > ind_b)
                    ind_c = ind_c_aux[ind_c]
                    if len(ind_c) > 1:
                        ind_c = [ind_c[0]]  # Adjusted by HJ
                    c2d = np.append(c2d, ind_c + start)
            # d point:
            if len(ind_c) != 0:
                ind_d = np.intersect1d(np.where(aux_m2d_ons < ind_e), np.where(aux_m2d_ons > ind_c))
                if len(ind_d) != 0:
                    ind_d_aux = aux_m2d_ons[ind_d]
                    ind_d, = np.where(ibi_2d_portion[ind_d_aux] == np.min(ibi_2d_portion[ind_d_aux]))
                    ind_d = ind_d_aux[ind_d]
                    if len(ind_d) != 0:
                        d2d = np.append(d2d, ind_d + start)
                else:
                    ind_d = ind_c
                    d2d = np.append(d2d, ind_d + start)
    a2d = a2d.astype(int)
    b2d = b2d.astype(int)
    c2d = c2d.astype(int)
    d2d = d2d.astype(int)
    e2d = e2d.astype(int)

    # Search in D3: P1 and P2 points
    p1p = np.empty(0)
    p2p = np.empty(0)
    for i in range(len(ons) - 1):
        start = ons[i]
        stop = ons[i + 1]
        ibi_portion = x[start:stop]
        ibi_1d_portion = d1x[start:stop]
        ibi_2d_portion = d2x[start:stop]
        ibi_3d_portion = d3x[start:stop]
        ind_b = np.intersect1d(np.where(b2d > start), np.where(b2d < stop))
        ind_b = b2d[ind_b]
        ind_c = np.intersect1d(np.where(c2d > start), np.where(c2d < stop))
        ind_c = c2d[ind_c]
        ind_d = np.intersect1d(np.where(d2d > start), np.where(d2d < stop))
        ind_d = d2d[ind_d]
        ind_dic = np.intersect1d(np.where(dic > start), np.where(dic < stop))
        ind_dic = dic[ind_dic]
        aux_p3d_pks, _ = sp.find_peaks(ibi_3d_portion)
        aux_p3d_ons, _ = sp.find_peaks(-ibi_3d_portion)
        # P1:
        if (len(aux_p3d_pks) != 0 and len(ind_b) != 0):
            ind_p1, = np.where(aux_p3d_pks > ind_b - start)
            if len(ind_p1) != 0:
                ind_p1 = aux_p3d_pks[ind_p1[0]]
                p1p = np.append(p1p, ind_p1 + start)
        # P2:
        if (len(aux_p3d_ons) != 0 and len(ind_c) != 0 and len(ind_d) != 0):
            if ind_c == ind_d:
                ind_p2, = np.where(aux_p3d_ons > ind_d - start)
                if len(ind_p2) == 0:  # Augmented by HJ
                    continue
                ind_p2 = aux_p3d_ons[ind_p2[0]]
            else:
                ind_p2, = np.where(aux_p3d_ons < ind_d - start)
                if len(ind_p2) == 0: # Augmented by HJ
                    continue
                ind_p2 = aux_p3d_ons[ind_p2[-1]]
            if len(ind_dic) != 0:
                aux_x_pks, _ = sp.find_peaks(ibi_portion)
                if ind_p2 > ind_dic - start:
                    ind_between = np.intersect1d(np.where(aux_x_pks < ind_p2),
                                                 np.where(aux_x_pks > ind_dic - start))
                else:
                    ind_between = np.intersect1d(np.where(aux_x_pks > ind_p2),
                                                 np.where(aux_x_pks < ind_dic - start))
                if len(ind_between) != 0:
                    ind_p2 = aux_x_pks[ind_between[0]]
            p2p = np.append(p2p, ind_p2 + start)
    p1p = p1p.astype(int)
    p2p = p2p.astype(int)

    off = ons[1:]
    ons = ons[:-1]
    if pks[0] < ons[0]:
        pks = pks[1:]
    if pks[-1] > off[-1]:
        pks = pks[:-1]

    # Creation of dictionary
    fidp = {'pks': pks.astype(int),
            'ons': ons.astype(int),
            'off': off.astype(int),
            'tip': tip.astype(int),
            'dia': dia.astype(int),
            'dic': dic.astype(int),
            'm1d': m1d.astype(int),
            'a2d': a2d.astype(int),
            'b2d': b2d.astype(int),
            'c2d': c2d.astype(int),
            'd2d': d2d.astype(int),
            'e2d': e2d.astype(int),
            'p1p': p1p.astype(int),
            'p2p': p2p.astype(int)
            }

    return fidp

\end{lstlisting}

\vspace{-0.3cm}
\subsection{Feature Extraction}
\label{subsec:code_fe}

\begin{lstlisting}[language=Python,label={lst:fe.py}, basicstyle=\scriptsize]
import numpy as np
from scipy.integrate import simps


def sys_dia_detection(fidp, data):
    # (From filtered data) Systolic BP = pks; Diastolic BP = dia
    sys = fidp['pks']
    dia = fidp['off']
    sys, dia = group_sys_dia(sys, dia)
    length = min(len(sys), len(dia))
    tss = np.zeros(length, dtype=int)
    tsd = np.zeros(length, dtype=int)
    sysv = np.zeros(length, dtype=float)
    diav = np.zeros(length, dtype=float)
    beat_no, adj_beat_no = 0, 0
    while beat_no < len(tss):
        sys_beat = data[sys[beat_no + adj_beat_no]]
        dia_beat = data[dia[beat_no + adj_beat_no]]
        if sys_beat >= dia_beat:
            tss[beat_no] = sys[beat_no + adj_beat_no]
            sysv[beat_no] = sys_beat
            tsd[beat_no] = dia[beat_no + adj_beat_no]
            diav[beat_no] = dia_beat
            beat_no += 1
        else:
            tss = np.delete(tss, beat_no)
            sysv = np.delete(sysv, beat_no)
            tsd = np.delete(tsd, beat_no)
            diav = np.delete(diav, beat_no)
            adj_beat_no += 1
    sys = np.column_stack((tss, sysv))
    dia = np.column_stack((tsd, diav))
    return sys, dia, tss, sysv, tsd, diav


def systolic_time_detection(f_pks, f_ons, pk_index, data, fs):
    # Systolic Time = Systolic peak - Onset
    value = (f_pks[pk_index] - f_ons[pk_index]) / fs
    # Systolic Area = Integral of PPG with limits [ons:pks]
    time_interval_start, time_interval_end = f_ons[pk_index], f_pks[pk_index]
    time_interval = np.arange(time_interval_start, time_interval_end)
    signal_interval = data[time_interval_start:time_interval_end]
    area = simps(signal_interval, time_interval)
    return value, area


def diastolic_time_detection(f_off, f_pks, pk_index, data, fs):
    # Diastolic Time = Offset - Systolic peak
    value = (f_off[pk_index] - f_pks[pk_index]) / fs
    # Diastolic Area = Integral of PPG with limits [pks:off]
    time_interval_start, time_interval_end = f_pks[pk_index], f_off[pk_index]
    time_interval = np.arange(time_interval_start, time_interval_end)
    signal_interval = data[time_interval_start:time_interval_end]
    area = simps(signal_interval, time_interval)
    return value, area


def delta_t_detection(f_dia, f_pks, pk_index, data, fs):
    # delta T = Dicrotic notch - Systolic peak
    value = (f_dia[pk_index] - f_pks[pk_index]) / fs
    # Delta Area = Integral of PPG with limits [pks:dia]
    time_interval_start, time_interval_end = f_pks[pk_index], f_dia[pk_index]
    time_interval = np.arange(time_interval_start, time_interval_end)
    signal_interval = data[time_interval_start:time_interval_end]
    abs_time_interval = np.arange(0, len(signal_interval))
    area = simps(signal_interval, abs_time_interval)
    return value, area


def pulse_area_detection(f_ons, f_off, pk_index, data):
    # Pulse Area = Integral of PPG with limits [ons:off]
    time_interval_start, time_interval_end = f_ons[pk_index], f_off[pk_index]
    time_interval = np.arange(time_interval_start, time_interval_end)
    signal_interval = data[time_interval_start:time_interval_end]
    value = simps(signal_interval, time_interval)
    return value


def resistive_index_detection(f_dia, f_pks, f_off, pk_index, data):
    # RI = (data[dia] - data[off]) / (data[pks] - data[off])
    h1 = data[f_dia[pk_index]] - data[f_off[pk_index]]
    h2 = data[f_pks[pk_index]] - data[f_off[pk_index]]
    value = h1 / h2
    return value


def vessel_volume_index_detection(f_pks, f_off, pk_index, data):
    # Vessel Volume Fill-Up (systolic) Index = data[pks] / max(data[all_systoles])
    max_v = max(data[f_pks[f_pks != 0]])
    v1 = data[f_pks[pk_index]] / max_v
    # Vessel Volume Drained (diastolic) Index = max(data[all_systoles]) / data[off]
    min_v = min(data[f_off[f_off != 0]])
    v2 = min_v / data[f_off[pk_index]]
    return v1, v2


def systolic_diastolic_width_detection(f_ons, f_pks, f_off, pk_index, data, fs, p):
    # Systolic width = data[(data[pks]-data[ons])*p/100]
    ind_ons, ind_pk, ind_off = f_ons[pk_index], f_pks[pk_index], f_off[pk_index]
    val_ons, val_pk, val_off = data[ind_ons], data[ind_pk], data[ind_off]
    threshold = p / 100 * (val_pk - val_ons) + val_ons
    sys_arr = np.abs(np.array(data[ind_ons:ind_pk]) - threshold)
    ind_sw = np.argmin(sys_arr)
    sw = (len(sys_arr) - 1 - ind_sw) / fs
    dia_arr = np.abs(np.array(data[ind_pk:ind_off]) - threshold)
    ind_dw = np.argmin(dia_arr)
    dw = (ind_dw - 0) / fs
    return sw, dw


def frequency_domain_features(signal, fs):
    # Compute the Fast Fourier Transform (FFT)
    fft_result = np.fft.fft(signal)
    # Compute the frequencies corresponding to the FFT result
    frequencies = np.fft.fftfreq(len(fft_result), 1 / fs)
    # Only consider positive frequencies
    positive_frequencies = frequencies[:len(frequencies) // 2]
    # Magnitude spectrum (absolute values of FFT result)
    magnitude_spectrum = np.abs(fft_result[:len(fft_result) // 2])
    # Find the index corresponding to the maximum magnitude
    peak_frequency_index = np.argmax(magnitude_spectrum)
    # Other frequency domain features
    mean_frequency = np.sum(positive_frequencies * magnitude_spectrum) / np.sum(magnitude_spectrum)
    total_power = np.sum(magnitude_spectrum)
    normalized_power_at_peak = magnitude_spectrum[peak_frequency_index] / total_power
    return {'mean_frequency': mean_frequency,
            'total_power': total_power,
            'normalized_power_at_peak': normalized_power_at_peak }
\end{lstlisting}

\subsection{Machine Learning}
\label{subsec:code_ml}

\begin{lstlisting}[language=Python,label={lst:ml.py}, basicstyle=\scriptsize]
import torch
import torch.nn as nn
import sklearn


class LinearRegression(nn.Module):
    def __init__(self, input_size, output_size, feature_importances):
        super(LinearRegression, self).__init__()
        self.linear = nn.Linear(input_size, output_size)
        self.feature_importances = feature_importances

    def forward(self, x):
        adjusted_input = x * self.feature_importances.unsqueeze(0)
        return self.linear(adjusted_input)


class NeuralNet(nn.Module):
    def __init__(self, input_size, hidden_size, output_size, feature_importances):
        super(NeuralNet, self).__init__()
        self.input_size = input_size
        self.l1 = nn.Linear(input_size, hidden_size)
        self.relu = nn.ReLU()
        self.l2 = nn.Linear(hidden_size, output_size)
        self.feature_importances = feature_importances

    def forward(self, x):
        adjusted_input = x * self.feature_importances.unsqueeze(0)
        out = self.l1(adjusted_input)
        out = self.relu(out)
        out = self.l2(out)
        return out


class LSTM(nn.Module):
    def __init__(self, input_size, hidden_size, out_features, feature_importances):
        super(LSTM, self).__init__()
        self.hidden_size = hidden_size
        self.lstm1 = nn.LSTMCell(input_size, hidden_size)
        self.lstm2 = nn.LSTMCell(hidden_size, hidden_size)
        self.linear = nn.Linear(hidden_size, out_features)
        self.feature_importances = feature_importances

    def forward(self, x):
        h_t = torch.zeros(x.size(0), self.hidden_size, dtype=torch.float32)
        c_t = torch.zeros(x.size(0), self.hidden_size, dtype=torch.float32)
        h_t2 = torch.zeros(x.size(0), self.hidden_size, dtype=torch.float32)
        c_t2 = torch.zeros(x.size(0), self.hidden_size, dtype=torch.float32)
        adjusted_input = x * self.feature_importances.unsqueeze(0)
        h_t, c_t = self.lstm1(adjusted_input, (h_t, c_t))
        h_t2, c_t2 = self.lstm2(h_t, (h_t2, c_t2))
        output = self.linear(h_t2)
        return output


class BiLSTM(nn.Module):
    def __init__(self, input_size, hidden_size, out_features, feature_importances):
        super(BiLSTM, self).__init__()
        self.hidden_size = hidden_size
        self.lstm = nn.LSTM(input_size, hidden_size, bidirectional=True)
        self.linear = nn.Linear(hidden_size * 2, out_features)
        self.feature_importances = feature_importances

    def forward(self, x):
        h_0 = torch.zeros(2, x.size(0), self.hidden_size, dtype=torch.float32)
        c_0 = torch.zeros(2, x.size(0), self.hidden_size, dtype=torch.float32)
        adjusted_input = x.unsqueeze(0) * self.feature_importances.unsqueeze(0)
        lstm_out, _ = self.lstm(adjusted_input, (h_0, c_0))
        output = self.linear(lstm_out[-1])
        return output


class GRU(nn.Module):
    def __init__(self, input_size, hidden_size, out_features, feature_importances):
        super(GRU, self).__init__()
        self.hidden_size = hidden_size
        self.gru1 = nn.GRUCell(input_size, hidden_size)
        self.gru2 = nn.GRUCell(hidden_size, hidden_size)
        self.linear = nn.Linear(hidden_size, out_features)
        self.feature_importances = feature_importances

    def forward(self, x):
        h_t = torch.zeros(x.size(0), self.hidden_size, dtype=torch.float32)
        h_t2 = torch.zeros(x.size(0), self.hidden_size, dtype=torch.float32)
        adjusted_input = x * self.feature_importances.unsqueeze(0)
        h_t = self.gru1(adjusted_input, h_t)
        h_t2 = self.gru2(h_t, h_t2)
        output = self.linear(h_t2)
        return output


class BiGRU(nn.Module):
    def __init__(self, input_size, hidden_size, out_features, feature_importances):
        super(BiGRU, self).__init__()
        self.hidden_size = hidden_size
        self.gru = nn.GRU(input_size, hidden_size, bidirectional=True)
        self.linear = nn.Linear(hidden_size * 2, out_features)
        self.feature_importances = feature_importances

    def forward(self, x):
        h0 = torch.zeros(2, x.size(0), self.hidden_size, dtype=torch.float32)
        adjusted_input = x.unsqueeze(0) * self.feature_importances.unsqueeze(0)
        out, _ = self.gru(adjusted_input, h0)
        out = out.squeeze(0)
        out = self.linear(out)
        return out

# Random Forest Regressor
rf_model = sklearn.ensemble.RandomForestRegressor(n_estimators=15)
\end{lstlisting}