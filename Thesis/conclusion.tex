The outcomes of this study on the application of various signal processing and machine learning approaches for BP prediction from PPG waveform data have yielded results
that provide valuable insights into the challenges and potential of this technology.
While the initial anticipation leaned towards achieving high accuracy levels, the performance of the models fell short of this mark.
Through rigorous experimentation with pre-processing techniques, feature extraction methods, and advanced ML architectures,
a deeper understanding of the complexities inherent in BP prediction from PPG waveforms has been gained.

In light of the limitations discussed, there are several avenues for improvement that require exploration.
The integration of both ECG and PPG data holds promise for enhancing prediction precision, as it can provide a more comprehensive view of cardiovascular dynamics.
Further advancements in signal processing approaches, such as adaptive filtering and noise reduction techniques, coupled with manual waveform filtering,
may substantially improve the quality of input data for the ML models.
Additionally, expanding the range of ML algorithms, investigating different median intervals for feature extraction,
and exploring alternative data splitting strategies are areas ripe for further investigation.

Despite the focus on ICU patient data in this study, the potential for expansion into ambulatory patients or real-time healthy patient monitoring presents exciting possibilities.
The identified limitations serve as guideposts for future research endeavors, urging a comprehensive examination of these areas to enhance the predictive capabilities of BP prediction models.

The overarching conclusion drawn from this study is the acknowledgment of the formidable challenge in creating a universal BP prediction model from PPG technology alone.
The observed trend where prediction accuracy improves with subject size, underscores the personalized nature of PPG waveforms.
Each individual's physiological characteristics, which evolve over time, contribute significantly to the variability in BP predictions.
To address this, the introduction of calibration techniques, incorporation of verifiable patient data for model training,
or the derivation of subject-specific parameters (e.g., age, body weight, height) are potential paths forward.
These considerations are crucial for the development of more nuanced and effective BP prediction models tailored to individual patients.

In conclusion, this study lays the foundation for future research endeavors aiming to refine the use of PPG technology for BP prediction.
By acknowledging the complexities and limitations while proposing avenues for improvement,
the study contributes to the ongoing dialogue surrounding personalized healthcare and the integration of advanced technologies into clinical practice.
