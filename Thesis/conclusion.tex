The findings of this study provide insights into the predictive performance of Signal Processing and Machine Learning models for BP prediction from PPG waveform data.

The study exceeded the minimum patient evaluations, yet the models' performance fell short of the desired accuracy levels.

% Outlook

Possible Improvements:
\begin{itemize}
    \item Integrate both the ECG and PPG graphs for a more precise prediction.
    \item Use further signal processing approaches.
    \item Manually filter waveform graphics (e.g., faulty PPG signal).
    \item Expand on machine learning algorithms.
    \item Experiment with different median intervals.
    \item Experiment with further data splitting approaches.
\end{itemize}

In light of all mentioned limitations in the discussion section, several possible improvements can be considered.
Integrating both ECG and PPG graphs could enhance prediction precision, while further signal processing approaches and manual waveform filtering may improve data quality.
Additionally, expanding on machine learning algorithms, experimenting with different median intervals, and exploring alternative data splitting approaches could lead to better model performance.

Despite the study's focus on ICU patient data, there is potential for expansion to include ambulatory patients or real-time healthy patient monitoring.
Future research should consider the identified limitations and explore the suggested improvements to enhance the predictive capabilities of the models in BP prediction tasks.

The general conclusion of the study is, that it is highly unlikely, to create a universal BP prediction model from the PPG technology.
As the subject size increases, the prediction accuracy increases, implying that PPG waveforms are very personal and might differ even as a subject ages.
To deal with this, possible calibration techniques have to be introduced, to train new ML models on verifiable patient data, or at least to find subject specific parameters,
such as age, body weight and height, or other distinct physiological features.

This study focused on ICU patient data, but it can prospectively be expanded on also ambulatory patients or even real-time healthy patient monitoring.
